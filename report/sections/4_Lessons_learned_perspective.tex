\section{Lessons Learned Perspective}

\subsection{Evolution and Refactoring}
The ideal apporach to refactoring the old application would be to analyse, divide and concur the old application into 
subtask and define the overall design of the new application and then start the refactoring process. However, we started
refactoring without having an idea of how the result/end system should be. Thus, the process became very unstructered. 
We have learned that this could have been avoided by better planning aided by eg. a Kanban board. This caused the 
refactoring process to take longer than anticipated and left us behind schedule. An example can be seen in this
\href{https://github.com/organizationGB/DevOps/commit/7bbccc97d6d69e90724b00e93e92334210490085}{commit} which is from a 
branch we ended up not using. \\

After refactoring the evolution of our application started. Our problem in regards to planning persisted, we did manage
to devide tasks but aggregating them afterwards could be challenging.  



Retrospective we should have been more structural in terms of e.g. having a daily meeting every time we met such 
that we knew things like, what are you working on, what do you expect and how far are we in the total process of 
reaching the weeks goal.


\subsection{Operation}

\subsection{Maintenance}

\subsection{DevOps style of work}